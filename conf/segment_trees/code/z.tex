\begin{itemize}
    \item{
        Llamaremos segmentos coincidentes (SC) a aquellas subcadenas que 
        coinciden con un prefijo de la cadena $S$.
    }
    \item{Mantenemos un intervalo $[l, r]$ que representa el segmento 
        coincidente que termina mas a la derecha. El \'indice $r$ puede 
        verse tambien como el l\'imite hasta el cual la cadena ha sido 
        explorada.
    }

    \item{
        Hacemos un ciclo de $1$ a $n$ ($n$, largo de la cadena). Para 
        cada posicion $i$ tenemos dos casos:
        \begin{itemize}
            \item{
                Si $i>r$: La posicion actual esta fuera de los limites 
                de lo que ya hemos procesado. Calcularemos $z[i]$ comparando 
                los valores uno por uno. Al final, si $z [ i ] >0$ tenemos 
                que actulizar los limites de $[ l ,r ]$ porque esta garantizado 
                que el nuevo $r =i+ z [ i ] - 1$ es mejor que el $r$ actual. 
                El limite $l$ se actualiza a la posicion actual.
            }
            \item{
                Si $i \leq r$: la posicion actual esta dentro de los limites 
                del actual SC. Podemos usar los valores de $z$ ya calculados.
                Observese que las subcadenas $s [ l... r ]$ y $s [ 0... r - l ]$ 
                coinciden. Esto significa que podemos tomar como una aproximacion 
                inicial para $z [ i ]$, el valor correspondiente en el segmento 
                ya calculado. Este valor es $z [ i - l ]$ , aunque puede ser 
                demasiado largo porque cuando se aplique a la posicion $i$, 
                puede exceder el indice $r$ y no sabemos nada de los caracteres 
                m\'as alla de $r$. Por lo tanto, tomamos 
                $min ( r - i+1, z [ i - l ] )$ . Luego de inicializar $z [ i ]$ 
                tratamos de coincidir los caracteres en las posiciones $z [ i ]$ 
                y $i+ z [ i ]$ uno por uno.
            }
        \end{itemize}    
    }
\end{itemize}
