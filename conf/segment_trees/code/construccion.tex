Antes de construir un segment tree, debemos decidir dos cosas:
\begin{itemize}
    \item{
        El \textit{valor} que se guarda en cada nodo. Por ejemplo, para un
        segment tree de suma, guardamos la suma en el rango correspondiente al nodo.
    }
    \item{
        La operaci\'on que mezcla dos nodos hijos para obtener el resultado del
        nodo padre. En el caso de la suma, se puede simplemente sumar los resultados de los hijos
        para obtener la suma en el padre.
    }
\end{itemize}

Decidir estos dos elementos puede parecer trivial pero, en usos m\'as avanzados de la
estructura, no lo es.

Para construir el segment tree utilizamos una funci\'on recursiva \texttt{build()},
comenzamos en la ra\'iz del arbol (nodo $1$ en el arreglo).
Si el intervalo del nodo actual cubre m\'as de un elemento, entonces podemos calcular su resultado
como la combinaci\'on de los resultados de sus hijos (en el caso del segment tree de suma, la suma;
en el de m\'inimo, el m\'inimo de los dos resultados de los hijos). Ahora llamamos recursivamente a
la funci\'on \texttt{build()} para calcular el resultado de los hijos. Si llegamos a un nodo cuyo
intervalo solo contiene una posicion (un nodo hoja), directamente le asignamos el resultado al nodo
y lo devolvemos.

La complejidad temporal de este procedimiento es $O(n)$. Cada nodo se visita una vez y la cantidad
de nodos en el segment tree es a lo m\'as $2\cdot n -1$.
